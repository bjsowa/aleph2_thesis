\documentclass[english,inz,shortabstract]{iithesis}

\usepackage[utf8]{inputenc}
\usepackage{natbib}
\usepackage[nottoc,notlot,notlof]{tocbibind}

\englishtitle   {A software for the Aleph2 robot -\fmlinebreak a Mars rover prototype}
\polishtitle    {Oprogramowanie do robota Aleph2 -\fmlinebreak prototypu łazika marsjańskiego}
\englishabstract{\ldots}
\polishabstract {\ldots}
\author         {Błażej Sowa}
\advisor        {dr Marek Materzok}
%\date          {}

\begin{document}

\chapter{Introduction}

\section{Motivation}
Trochę o łaziku, co skłoniło mnie do napisania oprogramowania, co było złego w poprzednim

\section{Goals}
Jakie docelowo miało być to oprogramowanie, w czym miało być lepsze od poprzedniego, jakie wymagania niefunkcjonalne ma spełniać

\section{Used technologies} % nie wiem jak to nazwać lepiej
Z jakich programów/systemów/magistrali korzysta moje oprogramowanie. Opis każdego z nich na tyle dokładny, żeby ktoś kto tego nie zna, zrozumiał mniej więcej wszystko o czym będę pisał później
\subsection{ROS}
\subsection{ROS control}
\subsection{Gazebo}
\subsection{RUBI}

\chapter{Design}
\section{Overview of provided ROS packages}


\bibliographystyle{plain}
\bibliography{bibliography}

\end{document}
